%%------------------------------------------------------
% 
%% UNIVERSIDADE FEDERAL DE SANTA CATARINA - UFSC
%
%% Prof.: Wyllian B. da Silva
%%
%% Template: estilo IEEEtran [paper com duas colunas]
%% Adaptado de: https://ieeeauthorcenter.ieee.org/create-your-ieee-article/use-authoring-tools-and-ieee-article-templates/ieee-article-templates/templates-for-transactions/
%               https://ctan.org/tex-archive/macros/latex/contrib/IEEEtran?lang=en

%% Instruções: http://mirrors.ctan.org/macros/latex/contrib/IEEEtran/IEEEtran_HOWTO.pdf
%%
%% Recomendações:
%% Utilize o Editor Kile (SO Linux)
%% Certifique-se de que a codificação de caracteres utilizada é a UTF-8





\documentclass[journal]{IEEEtran}


%%------------------------------------------------------
%% Packages
%%------------------------------------------------------
\usepackage[T1]{fontenc}           %% Codificação de caracteres
\usepackage[utf8]{inputenc}        %% Codificação de caracteres (conversão automática dos acentos)
\usepackage[dvips]{graphicx}       %% para a macro includegraphics 
\usepackage[english,brazil]{babel} %% PT_BR e EN (o último define a prioridade no arquivo)
\usepackage{pgf}                   %% macro para criar gráficos
\usepackage{epsfig}                %% or use the epsfig package if you prefer to use the old commands
\usepackage{graphics}              %% use the graphics package for simple commands
\usepackage{graphicx}              %% or use the graphicx package for more complicated commands
\usepackage{epstopdf}              %% enable EPS (convert to PDF)
\usepackage{float}                 %% float environment
\usepackage{eqparbox}              %% to define a group of boxes 
\usepackage{hyphenat}              %% prevent hyphenation
\usepackage{hyperref}              %% enalbe one-click link

% \usepackage{showframe} %% just for the example

% \usepackage[sort,compress]{cite} %% disable if natbib package is activated
\usepackage[numbers,sort&compress,square]{natbib} %% e.g., [2-5]



%%------------------------------------------------------
%% Definitions
%%------------------------------------------------------

\hyphenation{op-tical net-works semi-conduc-tor}

%% Can use something like this to put references on a page
%% by themselves when using endfloat and the captionsoff option.
\ifCLASSOPTIONcaptionsoff
  \newpage
\fi


%%----------------- Definindo as variáveis com números
\makeatletter
%
\newcommand{\prenome}{\afterassignment\prenome@aux\count0=}
\newcommand{\prenome@aux}{\csname prenome\the\count0\endcsname}
%
\newcommand{\nomedomeio}{\afterassignment\nomedomeio@aux\count0=}
\newcommand{\nomedomeio@aux}{\csname nomedomeio\the\count0\endcsname}
%
\newcommand{\sobrenome}{\afterassignment\sobrenome@aux\count0=}
\newcommand{\sobrenome@aux}{\csname sobrenome\the\count0\endcsname}
\makeatother
%%%%%

%%----------------- Configurações de hyperlinks
%% Não decorado, sem destaque
\hypersetup{
  colorlinks=false,
  pdfborder={0 0 0},
}




%%------------------------------------------------------
%% Configurações
%%------------------------------------------------------

%%----------------- Título
\title                                                {Template Para Escrita de Paper no Formato IEEE Journals, Utilizando a Classe IEEEtran.cls}

\newcommand{\emailautor}                              {seuemail@grad.ufsc.br}

\newcommand{\siglaRevista}                            {UFSC}

\newcommand{\Revista}                                 {Universidade Federal de Santa Catarina (UFSC)}



%%----------------- Autor Principal (a acenturação deverá ser indireta)
\newcommand{\prenomePrincipal}                        {Paul}
\newcommand{\nomedomeioPrincipal}                     {Adrien Maurice}
\newcommand{\sobrenomePrincipal}                      {Dirac}


%%----------------- Demais Autores
%% Segundo autor (a acenturação deverá ser indireta)
\expandafter\newcommand\csname prenome2\endcsname     {Niels}
\expandafter\newcommand\csname nomedomeio2\endcsname  {Henrik David}
\expandafter\newcommand\csname sobrenome2\endcsname   {Bohr}


%% Terceiro autor (a acenturação deverá ser indireta)
\expandafter\newcommand\csname prenome3\endcsname     {Cesare}
\expandafter\newcommand\csname nomedomeio3\endcsname  {Mansueto Giulio}
\expandafter\newcommand\csname sobrenome3\endcsname   {Lattes}

%% Quarto autor (a acenturação deverá ser indireta)
\expandafter\newcommand\csname prenome4\endcsname     {Erwin}
\expandafter\newcommand\csname nomedomeio4\endcsname  {Rudolf Josef Alexander}
\expandafter\newcommand\csname sobrenome4\endcsname   {Schr\"odinger}

%% Quinto autor (a acenturação deverá ser indireta)
\expandafter\newcommand\csname prenome5\endcsname     {Werner}
\expandafter\newcommand\csname nomedomeio5\endcsname  {Karl}
\expandafter\newcommand\csname sobrenome5\endcsname   {Heisenberg}




%%------------------------------------------------------
%% Autor(es)


%%----------------- Apenas um autor
\author{\IEEEauthorblockN{\prenomePrincipal~\nomedomeioPrincipal~\sobrenomePrincipal\IEEEauthorrefmark{1}}

\IEEEauthorblockA{\IEEEauthorrefmark{1}Universidade Federal de Santa Catarina (UFSC)}% <-this % stops an unwanted space
%%
\thanks{\Revista. Correspond\^encia ao autor: \prenomePrincipal~\nomedomeioPrincipal~\sobrenomePrincipal~(email: \emailautor).}}



% %%----------------- Vários Autores
% \author{\IEEEauthorblockN{\prenomePrincipal~\nomedomeioPrincipal~\sobrenomePrincipal\IEEEauthorrefmark{1}, 
% \prenome2~\nomedomeio2~\sobrenome2\IEEEauthorrefmark{2}, 
% \prenome3~\nomedomeio3~\sobrenome3\IEEEauthorrefmark{3}, 
% \prenome4~\nomedomeio4~\sobrenome4\IEEEauthorrefmark{3}, and 
% \prenome5~\nomedomeio5~\sobrenome5\IEEEauthorrefmark{4},~\IEEEmembership{Fellow,~IEEE}}
% 
% \IEEEauthorblockA{\IEEEauthorrefmark{1}Universidade Federal de Santa Catarina (UFSC)}
% 
% \IEEEauthorblockA{\IEEEauthorrefmark{2}School of Electrical and Computer Engineering, Georgia Institute of Technology, Atlanta, GA 30332 USA}
% 
% \IEEEauthorblockA{\IEEEauthorrefmark{3}Starfleet Academy, San Francisco, CA 96678 USA}
% 
% \IEEEauthorblockA{\IEEEauthorrefmark{4}Tyrell Inc., 123 Replicant Street, Los Angeles, CA 90210 USA}% <-this % stops an unwanted space
% %%
% \thanks{\Revista~(\siglaRevista). Correspond\^encia ao autor: \prenomePrincipal~\nomedomeioPrincipal~\sobrenomePrincipal~(email: \emailautor).}}



%%------------------------------------------------------
%% Cabeçalho
\markboth{\MakeUppercase{\Revista}}%% acentuação indireta
%% Apenas um autor:
{\sobrenomePrincipal: \MakeUppercase{\Revista}}
%% Mais de um autor:
% {\sobrenomePrincipal: \MakeLowercase{\textit{et al.}}: \Revista}


%%------------------------------------------------------
%% Abstract
\IEEEtitleabstractindextext{

  {\selectlanguage{brazil}
    \begin{abstract}
    Escreva aqui o seu resumo.
    \end{abstract}
    %%----------------- Keywords
    \renewcommand\IEEEkeywordsname{Palavras-chave} %% Palavras-chave ao invés de 'Index Terms'
    \begin{IEEEkeywords}
    Redes Sem Fio, QoS, QoE, PDI, Avalição Objetiva de Qualidade de Imagem e Vídeo, Sistemas de Controle, Sistemas de Automação, Propagação e Ondas, Eletrônica de Potência, Compatibilidade Eletromagnética, Sensores e Atuadores.
    \end{IEEEkeywords}
  }

  {\selectlanguage{english}
    \begin{abstract}
    The abstract goes here.
    \end{abstract}
    %%----------------- Keywords
    \begin{IEEEkeywords}
    Wireless Networks, QoS, QoE, PDI, Objective Image and Video Quality Assessment , Control Systems, Automation System, Wave Propagation, Power Electronics, Electromagnetic Compatibility, Sensors and Actuators.
    \end{IEEEkeywords}
  }
}








\begin{document}



%%------------------------------------------------------
%% Inserção de informações
\maketitle
\IEEEdisplaynontitleabstractindextext
\IEEEpeerreviewmaketitle


%%------------------------------------------------------
%% Section
\section{Introdução}

\IEEEPARstart{E}{ste} é um \textit{template} adaptado para publicação de trabalhos na \siglaRevista, cujo intuito é introduzir orientações na produção de \textit{papers} no formato IEEE \textsc{Transactions}, utilizando a linguagem de marcação de texto \LaTeX, por meio da classe \nohyphens{IEEEtran.cls} versão 1.8b ou posterior. Recomenda-se utilizar o editor {\ttfamily\fontsize{9}{9}\selectfont\texttt{Kile}} (Sistema Operacional Linux), configurado com o tipo de codificação binária Unicode {\ttfamily\fontsize{9}{9}\selectfont\texttt{UTF-8}} (\textit{8-bit Unicode Transformation Format}). Escreva no mínimo quatro (4) páginas e, no máximo, seis (6) páginas, excluindo-se o(s) apêndice(s).

A Figura~\ref{fig:fig_exemple} é um simples exemplo, utilizando inserção de figura, cuja dimensão da figura está em função da largura da coluna, por meio do parâmetro {\ttfamily\fontsize{9}{9}\selectfont\texttt{\string\columnwidth}}, ajustado em 23\%, \textit{i.e.}, ``{\ttfamily\fontsize{9}{9}\selectfont\texttt{width=.23\string\columnwidth}}''. Para uma melhor organização, a pasta ``{\ttfamily\fontsize{9}{9}\selectfont\texttt{figs}}'' será utilizada para armazenar as figuras. Os parâmetros {\ttfamily\fontsize{9}{9}\selectfont\texttt{!htbp}}, os quais significam aqui (h), topo (t), embaixo (b) e página com objetos flutuantes (p), respectivamente. Este parâmetros são utilizados segundo uma ordem de posicionamento da figura, neste caso, da esquerda para a direita, em que o símbolo ``{\ttfamily\fontsize{9}{9}\selectfont\texttt{!}}'' força a prioridade neste sentido.
\begin{figure}[!htbp]
\centering
\includegraphics[width=.23\columnwidth]{figs/brasao_UFSC_vertical_sigla.pdf}
\caption{Escreva a legenda da figura aqui.}
\label{fig:fig_exemple}
\end{figure}

Crie figuras em alta qualidade, com no mínimo 300 DPI (\textit{Dots Per Inch}). Recomenda-se utilizar o formato PDF (\textit{Portable Document Format}), ao invés de formatos comprimidos.

A Tabela~\ref{tab:exemplo} é um simples exemplo de tabela construída no ambiente {\ttfamily\fontsize{9}{9}\selectfont\texttt{table}}. Aqui também é possível utilizar os parâmetros {\ttfamily\fontsize{9}{9}\selectfont\texttt{!htbp}}, tal qual fora descrito para o ambiente {\ttfamily\fontsize{9}{9}\selectfont\texttt{figure}}.
\begin{table}[!htbp]
\renewcommand{\arraystretch}{1.3}
\caption{Exemplo de uma Tabela}
\label{tab:exemplo}
\centering
\begin{tabular}{|c||c|}
\hline
Um & Dois\\
\hline
Três & Quatro\\
\hline
\end{tabular}
\end{table}

Há outras maneiras mais elaboradas de criar tabelas em \LaTeX.

Este é um exemplo de citação com uma única referência \cite{Spiegel:1998} e com duas citações \cite{SILVA:2013,Oliveira:2016}. A seguir, segue um exemplo com múltiplas citações \cite{Einstein:1917,Lattes:1947,Ahmed:1974}. Observe que esta última citação está na forma compacta e ordenada, pois o pacote {\ttfamily\fontsize{9}{9}\selectfont\texttt{natbib}} foi configurado com os parâmetros ``{\ttfamily\fontsize{9}{9}\selectfont\texttt{numbers,sort\&compress,square}}''. Estilos de referências podem ser encontrados em \href{https://verbosus.com/bibtex-style-examples.html}{https://verbosus.com/bibtex-style-examples.html} \cite{article:1993,book:1993,booklet:1993,conference:1993,inbook:1993,incollection:1993,manual:1993,mastersthesis:1993,misc:1993,phdthesis:1993,proceedings:1993,techreport:1993,unpublished:1993,patent:1993}. Implemente o acesso a URLs com um clique por meio da macro {\ttfamily\fontsize{9}{9}\selectfont\texttt{\string\href}}, \textit{e.g.}, {\ttfamily\fontsize{9}{9}\selectfont\texttt{\string\href\{http://ufsc.br\}\{http://ufsc.br\}}}. Há duas formas de escrever suas referências: (i) construindo o arquivo {\ttfamily\fontsize{9}{9}\selectfont\texttt{bib}}; (ii) conteúdo de referência incorporado ao arquivo \TeX. A primeira opção pode ser implementada utilizando um editor de referências, \textit{e.g.}, JabRef (\href{http://www.jabref.org}{http://www.jabref.org}). Uma vez que o arquivo {\ttfamily\fontsize{9}{9}\selectfont\texttt{bib}} esteja editado, deve-se utilizar as macros a seguir para habilitar as referências citadas ao longo do texto.

{\fontsize{8}{8}\selectfont
\begin{verbatim}
\bibliographystyle{IEEEtran} % Estilo da referência
\bibliography{references}    % Caminho do arquivo
                             %   (sem extensão)
\end{verbatim}
}

A segunda opção deve ser posicionada no final do arquivo \TeX, geralmente, após a conclusão ou apêndices. A seguir, apresenta-se um exemplo desse tipo de construção. O parâmetro {\ttfamily\fontsize{9}{9}\selectfont\texttt{10}} em {\ttfamily\fontsize{9}{9}\selectfont\texttt{\string\begin\{thebibliography\}\{10\}}}, é usado para orientar a configuração da margem deslocada à direita, o qual está relacionado com a quantidade de referências. 

{\fontsize{8}{8}\selectfont
\begin{verbatim}
\begin{thebibliography}{10} 
  \bibitem{Reinelt:1991}
    Reinelt, G. 1991. {\em The Traveling 
    Salesman -- Computational Solutions for TSP 
    Applications.} Berlin: Springer Verlag.

  \bibitem{Caprara:1997} 
    Caprara, A. 1997. Sorting by reversals is 
    difficult. In: {\em Proceedings of the 
    First Annual International Conference on 
    Computational Molecular Biology 
    (RECOMB 97),} New York: ACM.  pp. 75-83.

  \bibitem{Huynen:1998} 
    Huynen, M.~A. and Bork, P. 1998. Measuring 
    genome evolution. {\em Proceedings of the 
    National Academy of Sciences USA} 
    95:5849--5856.

  \bibitem{Kopka:1999}
    H.~Kopka and P.~W. Daly, \emph{A Guide to 
    \LaTeX}, 3rd~ed. Harlow, 
    England: Addison-Wesley, 1999.
\end{thebibliography}
\end{verbatim}
}

A partir da referência \cite{article:1993}, há diversos exemplos de estilos. \textbf{Observação:} \textit{os termos em negrito são usados para destacar o tipo de referência, não devem ser configurados dessa maneira}.

Além de seção ({\ttfamily\fontsize{9}{9}\selectfont\texttt{\string\section\{\}}}), na sequência são apresentados dois exemplos de subdivisões textuais: (i) subseção ({\ttfamily\fontsize{9}{9}\selectfont\texttt{\string\subsection\{\}}}) e (ii) subsubseção ({\ttfamily\fontsize{9}{9}\selectfont\texttt{\string\subsubsection\{\}}}). Ao utilizar este \textit{template}, implemente somente até o nível três (subsubseção).

%%------------------------------------------------------
%% Subsection
\subsection{Título da Subseção}
Texto da subseção.

%%------------------------------------------------------
%% Subsubsection
\subsubsection{Título da Subsubseção}
texto da subsubseção.

Evite concluir uma seção e suas subdivisões (subseção e subsubseção) de maneira abrupta com figuras, tabelas ou outros elementos gráficos, encerre-a de maneira textual. 

Apresente nesta seção os problemas encontrados na sua pesquisa e aponte (destaque) resumidamente as soluções encontradas (contribuições). Isto irá preparar e chamar a atenção do leitor quanto ao conteúdo que será abordado nas próximas seções.

Ainda nesta seção, faça uma descrição do ``Estado da Arte'' ou trabalhos relacionados com os métodos e implementações adotadas na sua pesquisa ou trabalho.

No final da seção, apresente sucintamente as seções restantes.





%%------------------------------------------------------
%% Section
\section{Metodologia}
Esta seção descreve os materiais e métodos ou ainda o referencial teórico abordado.


%%------------------------------------------------------
%% Section
\section{Resultados ou Resultados Experimentais ou Resultados Esperados}
Nesta seção são apresentados os resultados de simulação e/ou experimentas do seu trabalho. Caso não tenham realizados experimentos ou obtidos resultados concretos, pode-se descrever os resultados esperados.


%%------------------------------------------------------
%% Section
\section{Conclusão ou Conclusões}
Escreva a(s) sua(s) conclusão(ões) nesta seção.



%%------------------------------------------------------
%% Appedix (if necessary)
\appendices

%%----------------- Apêndice A
\section{Título do Apêndice}
Texto da seção. Utilize o apêndice somente quando for necessário.


%%----------------- Apêndice B
\section{Título do Apêndice}
Texto da seção do apêndice. 

%%----------------- Apêndice C
\section{} %% sem título
Texto da seção do apêndice. Observe que esta seção não apresenta título, cuja macro foi configurada como {\ttfamily\fontsize{9}{9}\selectfont\texttt{\string\section\{\}}}.



%%------------------------------------------------------
%% Section (no numbering, use section* for acknowledgment)
%% UFSC: no necessary
% \section*{Acknowledgment}
% The authors would like to thank...



%%------------------------------------------------------
%% References (Option 1): extern file
%% Edit with JabRef, for instance
\bibliographystyle{IEEEtran} %% Estilo da referência
\bibliography{references}    %% Caminho do arquivo (sem extensão)



% %%------------------------------------------------------
% %% References (Option 2): incorporeted
% 
% \begin{thebibliography}{10} 
% 
%   \bibitem{Kopka:1999}
%     H.~Kopka and P.~W. Daly, \emph{A Guide to
%     \LaTeX}, 3rd~ed. Harlow, 
%     England: Addison-Wesley, 1999.
% 
%   \bibitem{Huynen:1998} 
%     Huynen, M.~A. and Bork, P. 1998. Measuring 
%     genome evolution. {\em Proceedings of the 
%     National Academy of Sciences USA} 
%     95:5849--5856.
% 
%   \bibitem{Caprara:1997} 
%     Caprara, A. 1997. Sorting by reversals is 
%     difficult. In: {\em Proceedings of the 
%     First Annual International Conference on 
%     Computational Molecular Biology 
%     (RECOMB 97),} New York: ACM.  pp. 75-83.
% 
%   \bibitem{Reinelt:1991}
%     Reinelt, G. 1991. {\em The Traveling 
%     Salesman -- Computational Solutions for TSP 
%     Applications.} Berlin: Springer Verlag.
% 
% \end{thebibliography}



\end{document}
